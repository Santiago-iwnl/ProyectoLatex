\documentclass{article}
\usepackage[utf8]{inputenc}
\usepackage[spanish]{babel}
\usepackage{listings}
\usepackage{graphicx}
\graphicspath{ {images/} }
\usepackage{cite}

\begin{document}

\begin{titlepage}
    \begin{center}
        \vspace*{1cm}
            
        \Huge
        \textbf{Parcial 1 - Calistenia}
            
        \vspace{0.5cm}
        \LARGE
        Tarea 1
            
        \vspace{1.5cm}
            
        \textbf{Santiago Vélez Arboleda.}
            
        \vfill
            
        \vspace{0.8cm}
            
        \Large
        Despartamento de Ingeniería Electrónica y Telecomunicaciones\\
        Universidad de Antioquia\\
        Medellín\\
        Marzo de 2021
            
    \end{center}
\end{titlepage}

\tableofcontents
\newpage
\section{Introducción al problema.}\label{intro}
Se necesita  describir como llevar  dos objetos del estado A al estado B mediante una serie de pasos sucesivos detallados. Para realizar la solución del problema se estudia  mediante experimentos y tomando nota de los procesos que pueden llegar a la solución del problema. 

\section{Desarrollo del problema} \label{contenido}
 
\subsection{Contexto del problema.}
Se necesita llevar un par de tarjetas de un estado A a un estado B. El estado inicial A es aquel donde las tarjetas se encuentran debajo de una hoja, para seguidamente llevar las tarjetas a un estado B donde formarán un triángulo sobre la hoja de papel  empleando una sola mano.

\subsection{Solución del problema.}
Para dar solución al al anterior problema, se siguen los siguientes pasos: 


\begin{enumerate}
\item Sujetar la hoja y situarla en un lugar apartado de la mesa. 
\item Recoger las tarjetas con la mano dominante y         juntarlas de tal forma que ambas coincidan con su altura. 
\item 	Sujetar nuevamente la hoja con la mano no dominante para después situarla en el centro de la mesa.
\item  Posicionar las tarjetas sobre la hoja de tal forma que la base (la sección corta de las tarjetas) quede apuntando hacia el pecho. 
\item Con el dedo pulgar, posicionado en la esquina superior izquierda y el dedo índice posicionado en la esquina superior derecha sujetar ambas tarjetas formando una pinza con ambos dedos. 
\item 	 Poner las tarjetas de pie sobre la hoja.
\item Posicionar el dedo anular en la altura media de las tarjetas y seguidamente introducirlo entre el medio de las mismas sin dejar de sujetarlas en la parte superior.
\item 	Tratar de formar un ángulo de treinta grados aproximadamente entre las tarjetas para que ambas queden en equilibrio. 
\end{enumerate}
\subsection{Conclusiones}
"Para dar las instrucciones se deben de dar de forma precisa para realizar una tarea de la mejor manera. Es recomendable evitar ambigüedades y ser directo con los pasos a realizar debido que en el proceso de  elaboración  se pueden cometer errores que puede desencadenar en un problema mayor. "

- Wilson Vélez Garzón


Es importante seguir una secuencia de pasos ordenados para la realización de un trabajo. Esto puede evitar que se cometan errores y así mismo se puede optimizar tanto tiempo como recursos en un medio como la industria. "


- Maria Elena Arboleda Mora

"Para realizar una tarea se debe de conocer el cómo realizarla. Para ello se deben de estudiar e investigar los procesos debidos para la elaboración de la misma. Se necesita de pasos, referencias e indicaciones sobre el proceso de realización del trabajo con el fin de cumplir con los objetivos estipulados. "

- Maribel Arboleda Mora.

''Este ejercicio puede ser un ejemplo de cómo se realiza un algoritmo de programación. Como preliminar al proceso de elaboración del código, se debe de estudiar de forma minuciosa el problema en general. Dividir el nodo del problema en sub-nodos para después ordenarlos según su dificultad y así dar solución a cada uno de ellos. Seguidamente, realizar pruebas de escritorio de dichas soluciones para así mismo después buscar procesos que optimicen el algoritmo empleado. Así mismo, buscar los posibles errores que se presenten durante la ejecución de las órdenes para tener un código completo. ''

- Santiago Vélez Arboleda.










\end{document}
